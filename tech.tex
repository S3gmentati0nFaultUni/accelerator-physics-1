\documentclass[a4paper,10pt]{article}
\usepackage{src/preamble}

\begin{document}

\noindent
\begin{center}
	\textbf{{\Large MAGNETI SUPERCONDUTTORI}} \\
\end{center}

\noindent
\textbf{Autore: Alessandro Biagiotti} \textit{Università degli studi di Milano,
	Milano, Italia}
\\

\noindent
\textbf{INTRODUZIONE:}
\\
Come già introdotto nel documento dedicato alla parte teorica, in questo secondo documento mi
dedicherò a una trattazione più pratica delle tecnologie superconduttive che sono oggi più in
utilizzo nel campo della fisica delle alte energie, e non solo (basti pensare ad applicazioni
ingegneristiche per treni ad alta velocità \cite{maglev}).

I superconduttori sono materiali caratterizzati da un brusco annullamento della resistività ($R =
	0$ e non $R \approx 0$) una volta raggiunta una certa temperatura, nota come temperatura
critica.
\begin{figure}[h!]
	\centering

	\includegraphics[scale=0.35]{fig/The-evolution-of-critical-temperatures-since-the-discovery-of-superconductivity.png}
	\caption{
		Temperatura critica di varie leghe che sono state scoperte nel corso degli anni
		\cite{critical-temp}
	}
\end{figure}
Il comportamento dei superconduttori puri è stato spiegato tramite la teoria BCS, risalente al
1957, l'azzeramento della resistività del materiale è dovuto alla formazione delle cosiddette
coppie di Cooper.

Una coppia di Cooper è una coppia di elettroni che viaggiano insieme ed è generata da
un'interazione tra un elettrone e il reticolo cristallino (che è carico positivamente).
Quest'interazione porta a uno sbilanciamento della carica locale che passa da negativa a positiva
pertanto pertanto è possibile che un altro elettrone venga attratto, dando vita a una coppia di
Cooper \cite{cooper-cambridge}. Una spiegazione più dettagliata del fenomeno mette in relazione
questo comportamento con uno scambio di fononi \cite{quantum-springer}.

In un normale conduttore la resistenza elettrica sarebbe generata dallo scattering degli elettroni
dovuto a collisioni con i nuclei positivi del reticolo cristallino. All'interno di un
superconduttore, però, la maggioranza degli elettroni sono coppie di Cooper. Queste coppie, sebbene
interagiscano con il reticolo cristallino, non ricevono abbastanza energia per spezzare il legame
che le tiene insieme e, di fatto, non si dividono negli elettroni che le formano; pertanto non
vanno incontro allo scattering che caratterizza la normale resistenza elettrica dei conduttori
\cite{bcs-cambridge}.

Sebbene la mancanza di resistenza possa far pensare che un superconduttore sia in grado di
supportare una quantità infinita di tensione e densità di corrente si è scoperto, come si vedrà nel
seguito, che esistono altri limiti fisici che provocano la perdità della proprietà di
superconduttività.

All'interno di questo documento andrò ad esporre quali siano le tipologie di superconduttori
attualmente in esistenza e infine esplorerò alcune delle leghe più utilizzate e alcune tecnologie
più innovative nel campo della superconduzione a temperature elevate.

\bigskip
\phantomsection
\label{sec:type-one}
\noindent
\textbf{EFFETTO MEISSNER E TIPI DI SUPERCONDUTTORI:}
\\
Prima di poter introdurre la differenza tra superconduttori del primo e del secondo tipo bisogna
introdurre un'altra importante caratteristica dei superconduttori.

Questi, infatti, hanno un comportamento diamagnetico, il che significa che sono in grado di
espellere campi magnetici che normalmente li attraverserebbero. Questo comportamento, noto come
Meissner effect, fu scoperto nel 1933 da Meissner e Ochsenfeld\cite{meissner}. Per quanto visto in
precedenza la resistenza all'interno di un superconduttore a temperatura inferiore a $T_c$ è $0$,
questo significa che in presenza di una certa densità di corrente $\ve{J}$ il campo elettrico
all'interno del conduttore può essere calcolato come:
\begin{equation}
	\ve{E} = R \times \ve{J} = \ve{0}
\end{equation}
Quindi se consideriamo la legge di Ampere-Maxwell
\begin{equation*}
	\curl{E} = - \pder{\ve{B}}{t} = 0
\end{equation*}
Quanto illustrato sopra porterebbe quindi a dire che il campo magnetico all'interno del
superconduttore è costante e quindi il flusso sarà a sua volta costante.

In realtà il lavoro di Meissner ha dimostrato che il comportamento del superconduttore è ben
diverso, infatti l'espulsione del campo magnetico deriva dalla presenza di "correnti di
schermatura"\footnote{screening currents} sulla superficie del superconduttore che generano un
campo magnetico opposto a quello imposto sul conduttore, cancellandone
l'effetto\cite{super-fundamentals}. Ulteriori studi da parte dei fratelli London dimostrarono che,
in realtà vi è un certo livello di penetrazione del campo magnetico all'interno del conduttore, ma
l'intensità di quest'ultimo scala esponenzialmente con la distanza di penetrazione all'interno del
materiale\cite{ssp}.

I materiali superconduttori, sulla base di come si comportano in presenza di un campo magnetico
esterno, sono di tipo I o tipo II.

Tutti i materiali superconduttori di tipo I presentano una linea di demarcazione forte tra lo stato
di repulsione del campo magnetico (\emph{meissner state}) e lo stato di "permeazione" del campo
magnetico (\emph{normal state}). Questa
caratteristica è resa evidente da Figura \ref{fig:phase-diagram}
\begin{figure}[h!]
	\centering

	\includegraphics[scale=0.35]{fig/phase-diagram.jpg}
	\caption{
		Diagrammi di fase per superconduttori di tipo I e tipo
		II\cite{super-types}
	}\label{fig:phase-diagram}
\end{figure}

\bigskip
\phantomsection
\label{sec:metallurgy}
\noindent
\textbf{CENNI DI METALLURGIA:}

\clearpage

\printbibliography

\end{document}
