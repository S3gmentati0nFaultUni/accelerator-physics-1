\documentclass[a4paper,10pt]{article}
\usepackage{src/preamble}

\begin{document}

\noindent
\begin{center}
	\textbf{{\Large MAGNETI SUPERCONDUTTORI}} \\
\end{center}

\noindent
\textbf{Autore: Alessandro Biagiotti} \textit{Università degli studi di Milano,
	Milano, Italia}
\\

\noindent
\textbf{INTRODUZIONE:}
\\
Come già introdotto nel documento dedicato alla parte teorica, in questo secondo documento mi
dedicherò a una trattazione più pratica delle tecnologie superconduttive che sono oggi più in
utilizzo nel campo della fisica delle alte energie, e non solo (basti pensare ad applicazioni
ingegneristiche per treni ad alta velocità \cite{maglev}).

I superconduttori sono materiali caratterizzati da un brusco annullamento della resistività ($R =
	0$ e non $R \approx 0$) una volta raggiunta una certa temperatura, nota come temperatura
critica.
\begin{figure}[h!]
	\centering

	\includegraphics[scale=0.35]{fig/The-evolution-of-critical-temperatures-since-the-discovery-of-superconductivity.png}
	\caption{
		Temperatura critica di varie leghe che sono state scoperte nel corso degli anni
		\cite{critical-temp}
	}
\end{figure}
Il comportamento dei superconduttori puri è stato spiegato tramite la teoria BCS, risalente al
1957, l'azzeramento della resistività del materiale è dovuto alla formazione delle cosiddette
coppie di Cooper.

Una coppia di Cooper è una coppia di elettroni che viaggiano insieme ed è generata da
un'interazione tra un elettrone e il reticolo cristallino (che è carico positivamente).
Quest'interazione porta a uno sbilanciamento della carica locale che passa da negativa a positiva
pertanto pertanto è possibile che un altro elettrone venga attratto, dando vita a una coppia di
Cooper \cite{cooper-cambridge}. Una spiegazione più dettagliata del fenomeno mette in relazione
questo comportamento con uno scambio di fononi \cite{quantum-springer}.

In un normale conduttore la resistenza elettrica sarebbe generata dallo scattering degli elettroni
dovuto a collisioni con i nuclei positivi del reticolo cristallino. All'interno di un
superconduttore, però, la maggioranza degli elettroni sono coppie di Cooper. Queste coppie, sebbene
interagiscano con il reticolo cristallino, non ricevono abbastanza energia per spezzare il legame
che le tiene insieme e, di fatto, non si dividono negli elettroni che le formano; pertanto non
vanno incontro allo scattering che caratterizza la normale resistenza elettrica dei conduttori
\cite{bcs-cambridge}.

Sebbene la mancanza di resistenza possa far pensare che un superconduttore sia in grado di
supportare una quantità infinita di tensione e densità di corrente si è scoperto, come si vedrà nel
seguito, che esistono altri limiti fisici che provocano la perdità della proprietà di
superconduttività.

All'interno di questo documento andrò ad esporre quali siano le tipologie di superconduttori
attualmente in esistenza e infine esplorerò alcune delle leghe più utilizzate e alcune tecnologie
più innovative nel campo della superconduzione a temperature elevate.

\bigskip
\phantomsection
\makeatletter\def\@currentlabel{\texttt{(II)}}\makeatother
\label{sec:quench}
\noindent
\textbf{SUPERCONDUTTORI DI TIPO I:}

\bigskip
\phantomsection
\makeatletter\def\@currentlabel{\texttt{(III)}}\makeatother
\label{sec:mariotto}
\noindent
\textbf{SUPERCONDUTTORI DI TIPO II:}

\bigskip
\phantomsection
\makeatletter\def\@currentlabel{\texttt{(III)}}\makeatother
\label{sec:mariotto}
\noindent
\textbf{CENNI DI METALLURGIA:}

\bibliography{tech}

\clearpage

\end{document}
