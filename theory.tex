\documentclass[a4paper,10pt]{article}
\usepackage{src/preamble}

\begin{document}

\noindent
\begin{center}
	\textbf{{\Large QUENCH DI MAGNETI SUPERCONDUTTORI}} \\
\end{center}

\noindent
\textbf{Autore: Alessandro Biagiotti} \hfill \textit{Università degli studi di Milano, Milano, Italia}
\\

\noindent
\phantomsection
\makeatletter\def\@currentlabel{\texttt{(I)}}\makeatother
\label{sec:intro}
\textbf{INTRODUZIONE:}
\\
L'utilizzo di magneti superconduttori si rivela sempre più una necessità, soprattutto nel campo
della fisica nucleare e subnucleare; magneti normal-conduttori sono in grado di generare campi
nell'ordine di $\sim 1$T \cite{magnets}.

Per consentire un aumento dell'energia in vista di futuri acceleratori come FCC è necessario fare in
modo che il campo magnetico prodotto all'interno dell'acceleratore cresca di pari passo, altrimenti
saremmo costretti ad aumentare il raggio delle macchine in maniera considerevole per fare fronte
alla rigidità magnetica che cresce di pari passo con l'energia della macchina.

Ad oggi i dipoli superconduttori di LHC sono capaci di generare campi magnetici fino a $8.3$T
\cite{lhc-field}, un valore impensabile per qualsiasi magnete normal-conduttore.

Il problema dei magneti superconduttori è che necessitano di essere mantenuti a temperature
criogeniche per poter operare in uno stato di resistività $0$. Nel caso di LHC questi magneti sono
raffreddati tramite elio superfluido e quindi sono tenuti a temperature di circa $2$K. Eppure ciò
può non bastare, infatti spesso si verificano i cosiddetti fenomeni di \i{quench}.

Un fenomeno di quench può essere descritto come segue: Un'improvvisa, inattesa e inarrestabile
transizione allo stato normal-conduttore da parte di un superconduttore in operazione all'interno di
un dispositivo, che comporta la conversione dell'energia in esso contenuta in calore, il che può
portare alla distruzione del dispositivo stesso se non controllata \cite{quench-pres}.

Nel seguito cercherò di illustrare meglio cosa possa causare un fenomeno di quench
\ref{sec:quench} e una specifica tecnica di analisi per prevederne la posizione all'interno del
magnete superconduttore \ref{sec:mariotto}.

\bigskip
\phantomsection
\makeatletter\def\@currentlabel{\texttt{(II)}}\makeatother
\label{sec:quench}
\noindent
\textbf{FENOMENI DI QUENCH:}

\bigskip
\phantomsection
\makeatletter\def\@currentlabel{\texttt{(III)}}\makeatother
\label{sec:mariotto}
\noindent
\textbf{METODO DELL'ANALISI ARMONICA}

\clearpage

\bibliography{theory}

\end{document}
